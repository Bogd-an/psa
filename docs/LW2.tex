\documentclass[a4paper]{article}
% \documentclass[a4paper,14pt]{extarticle} 

% \usepackage[T2A]{fontenc}
% \usepackage[utf8]{inputenc}
% \usepackage[ukrainian]{babel}

% \usepackage{graphicx}
% \usepackage{geometry}
% \geometry{left=25mm, right=25mm, top=20mm, bottom=20mm}

% \renewcommand{\baselinestretch}{1.5}
% \setlength{\parindent}{3em}


\newcommand{\makrosTitle}[2]{
    \begin{titlepage}
        \centering
        \textbf{Міністерство освіти і науки України}\\
        \textbf{КИЇВСЬКИЙ ПОЛІТЕХНІЧНИЙ УНІВЕРССИТЕТ}\\[2cm]
        \raggedleft
        Кафедра автоматизації та систем неруйнівного контролю\\
        Група ПМ-11\\[2cm]
        \centering
        \textbf{ПРОЕКТУВАННЯ СИСТЕМ АВТОМАТИЗАЦІЇ}\\[1cm]
        \textbf{ЗВІТ З ЛАБОРАТОРНОЇ РОБОТИ №#1}\\[1cm]
        \textbf{#2}\\[3cm]
        \begin{flushleft}
            Керівник  \qquad\qquad\quad \hfill\qquad (підпис)\hfill 
            д.т.н., проф. Черепанська І. Ю.\\
            \hfill (дата)\\[2cm]
            Виконавець\hfill (підпис)\hfill Погорєлов Б. Ю.\\
            \hfill (дата)\\[2cm]
        \end{flushleft}
        \centering
        2025
    \end{titlepage}
}

% \begin{document}
    % \maketitlepage{автоматизації та систем неруйнівного контролю}{ПМ-11}
    % {XXXXXXXXXXXXXXXXXXXXXXXXXXXXXXXX}{X}{XXXX}
% \end{document}

\begin{document}
    \makrosLab{2}{ ПМ1109.04.00.01 ЛР}{
        Розробка та складання схем \\
        електричних принципових керування \\ 
        промисловими двигунами
    }

    \section*{Тема роботи}
    Розробка та складання схем електричних принципових керування
    промисловими двигунами

    \section*{Мета роботи}
    Вивчити будову та принцип дії промислових двигунів
    різних типів, як складових систем автоматичного
    керування / регулювання / контролю. Навчитися складати схеми електричні
    принципові для керування промисловими двигунами різних типів.

    \section*{Вихідні дані (Варіант 09)}
    \begin{table}[h!]
        \centering
        \begin{tabular}{|l|c|}
            \hline
            \textbf{Параметр} & \textbf{Значення} \\
            \hline
            Потужність, кВт & 1,0 \\
            \hline
            cos$\varphi$ & 0,86 \\
            \hline
            Швидкість обертання n ном, об/хв & 2850 \\
            \hline
            $\gamma$ (перенавантажувальна здатність) & 2,2 \\
            \hline
            ККД, \% & 91 \\
            \hline
            $\alpha$ (кратність пускового струму) & 5,1 \\
            \hline
            $\beta$ (кратність пускового моменту) & 2,35 \\
            \hline
        \end{tabular}
        \caption{Вихідні дані для розрахунків}
    \end{table}

    \section*{Розрахунки}
    \subsection*{Розрахунок споживаної потужності}
    \begin{equation}
        P_{\text{спож}} = \frac{P{\text{ном}}}{\eta} = \frac{1,0}{0,91} = 1,10 \text{ кВт}
    \end{equation}

    \subsection*{Розрахунок повної потужності}
    \begin{equation}
        S = \frac{P_{\text{спож}}}{\cos \varphi} = \frac{1,10}{0,86} = 1,28 \text{ кВА}
    \end{equation}

    \subsection*{Розрахунок струму}
    \begin{equation}
        I = \frac{S}{\sqrt{3} U} = \frac{1,28 \times 10^3}{\sqrt{3} \times 380} = 1,95 \text{ А}
    \end{equation}

    \subsection*{Розрахунок обертового моменту}
    \begin{equation}
        M = \frac{P_{\text{ном}} \times 60}{2 \pi n} = \frac{1,0 \times 60}{2 \pi \times 2850} = 3,34 \text{ Нм}
    \end{equation}

    \subsection*{Розрахунок пускового моменту}
    \begin{equation}
        M_{\text{пуск}} = \beta \times M_{\text{ном}} = 2,35 \times 3,34 = 7,84 \text{ Нм}
    \end{equation}

    \section*{Графік залежності обертового моменту від ковзання}
    % (Графік можна побудувати в MATLAB, Excel або Python, та вставити у форматі PNG через \texttt{\includegraphics{}}).

    \section*{Схеми підключення}
    % \includegraphics[width=0.8\textwidth]{circuit.png}

    \section*{Висновки}
    У ході роботи були проведені розрахунки параметрів трифазного асинхронного двигуна.
    Розраховані струми, потужності та моменти підтвердили можливість його використання у керованих системах.
    Також побудований графік залежності моменту від ковзання дозволив оцінити динамічні характеристики двигуна.

    \section*{Контрольні питання}
\begin{enumerate}
    \item Чому асинхронний двигун так називається? \\
    Асинхронний двигун називається так тому, що частота обертання його ротора не співпадає з частотою обертання магнітного поля статора (яка визначається частотою змінного струму). Різниця між цими частотами називається ковзанням.
    
    \item Чому є небажаною велика сила пускового струму? \\
    Велика сила пускового струму небажана, оскільки вона може призвести до значних механічних та електричних навантажень на двигун і мережу, викликати пошкодження ізоляції проводів, зменшити термін служби обладнання, а також викликати перевантаження трансформаторів і підстанцій.
    
    \item Що використовують для зниження сили пускового струму? \\
    Для зниження сили пускового струму використовують спеціальні пристрої, такі як стартери з обмеженням струму, трансформатори з регульованим напругою або пристрої плавного пуску, що забезпечують поступове збільшення напруги на двигуні.
\end{enumerate}


\end{document}
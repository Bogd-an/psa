\documentclass[a4paper]{article}
% \documentclass[a4paper,14pt]{extarticle} 

% \usepackage[T2A]{fontenc}
% \usepackage[utf8]{inputenc}
% \usepackage[ukrainian]{babel}

% \usepackage{graphicx}
% \usepackage{geometry}
% \geometry{left=25mm, right=25mm, top=20mm, bottom=20mm}

% \renewcommand{\baselinestretch}{1.5}
% \setlength{\parindent}{3em}


\newcommand{\makrosTitle}[2]{
    \begin{titlepage}
        \centering
        \textbf{Міністерство освіти і науки України}\\
        \textbf{КИЇВСЬКИЙ ПОЛІТЕХНІЧНИЙ УНІВЕРССИТЕТ}\\[2cm]
        \raggedleft
        Кафедра автоматизації та систем неруйнівного контролю\\
        Група ПМ-11\\[2cm]
        \centering
        \textbf{ПРОЕКТУВАННЯ СИСТЕМ АВТОМАТИЗАЦІЇ}\\[1cm]
        \textbf{ЗВІТ З ЛАБОРАТОРНОЇ РОБОТИ №#1}\\[1cm]
        \textbf{#2}\\[3cm]
        \begin{flushleft}
            Керівник  \qquad\qquad\quad \hfill\qquad (підпис)\hfill 
            д.т.н., проф. Черепанська І. Ю.\\
            \hfill (дата)\\[2cm]
            Виконавець\hfill (підпис)\hfill Погорєлов Б. Ю.\\
            \hfill (дата)\\[2cm]
        \end{flushleft}
        \centering
        2025
    \end{titlepage}
}

% \begin{document}
    % \maketitlepage{автоматизації та систем неруйнівного контролю}{ПМ-11}
    % {XXXXXXXXXXXXXXXXXXXXXXXXXXXXXXXX}{X}{XXXX}
% \end{document}

\begin{document}
    \makrosLab{2}{ ПМ1109.04.00.01 ЛР}{
        Розробка та складання схем \\
        електричних принципових керування \\ 
        промисловими двигунами
    }

    \section*{Тема роботи}
    Розробка та складання схем електричних принципових керування
    промисловими двигунами

    \section*{Мета роботи}
   
    Вивчити будову та принцип дії промислових двигунів
різних типів, як складових систем автоматичного
керування / регулювання / контролю. Навитися складати схеми електричні
принципові для керування промисловими двигунами різних типів.

    \section*{Хід роботи}

    \subsection*{Вихідні дані (Варіант 09)}
\begin{table}[h!]
    \centering
    \begin{tabular}{|l|c|}
        \hline
        \textbf{Параметр} & \textbf{Значення} \\
        \hline
        Потужність, кВт & 1,0 \\
        \hline
        cos$\varphi$ & 0,86 \\
        \hline
        % Швидкість обертання $n_{\text{ном}}$, об/хв & 2850 \\
        \hline
        $\gamma$ (перенавантажувальна здатність) & 2,2 \\
        \hline
        ККД, \% & 91 \\
        \hline
        $\alpha$ (кратність пускового струму) & 5,1 \\
        \hline
        $\beta$ (кратність пускового моменту) & 2,35 \\
        \hline
    \end{tabular}
    \caption{Вихідні дані для розрахунків}
\end{table}

\subsection*{Теоретичні відомості}
(Тут можна додати короткий опис принципу роботи асинхронного двигуна та його основні характеристики).

\subsection*{Розрахунки}
(Тут слід зробити необхідні розрахунки: потужність, споживаний струм, обертовий момент, пусковий момент тощо).

\subsection*{Схеми підключення}
(Можна вставити схеми підключення двигуна у трифазну та однофазну мережу, використовуючи команду `\includegraphics{}` для вставки зображень).

\subsection*{Графік залежності обертового моменту від ковзання}
(Якщо є можливість побудувати графік у MATLAB, Excel або Python, його можна зберегти у форматі PNG і вставити через `\includegraphics{}`).

\section*{Висновки}
(Формулювання основних висновків, отриманих у ході роботи).

\section*{Контрольні питання}
\begin{enumerate}
    \item Чому асинхронний двигун так називається?
    \item Чому є небажаною велика сила пускового струму?
    \item Що використовують для зниження сили пускового струму?
\end{enumerate}


\end{document}
\documentclass[a4paper]{article}
\usepackage{listings}
% \documentclass[a4paper,14pt]{extarticle} 

% \usepackage[T2A]{fontenc}
% \usepackage[utf8]{inputenc}
% \usepackage[ukrainian]{babel}

% \usepackage{graphicx}
% \usepackage{geometry}
% \geometry{left=25mm, right=25mm, top=20mm, bottom=20mm}

% \renewcommand{\baselinestretch}{1.5}
% \setlength{\parindent}{3em}


\newcommand{\makrosTitle}[2]{
    \begin{titlepage}
        \centering
        \textbf{Міністерство освіти і науки України}\\
        \textbf{КИЇВСЬКИЙ ПОЛІТЕХНІЧНИЙ УНІВЕРССИТЕТ}\\[2cm]
        \raggedleft
        Кафедра автоматизації та систем неруйнівного контролю\\
        Група ПМ-11\\[2cm]
        \centering
        \textbf{ПРОЕКТУВАННЯ СИСТЕМ АВТОМАТИЗАЦІЇ}\\[1cm]
        \textbf{ЗВІТ З ЛАБОРАТОРНОЇ РОБОТИ №#1}\\[1cm]
        \textbf{#2}\\[3cm]
        \begin{flushleft}
            Керівник  \qquad\qquad\quad \hfill\qquad (підпис)\hfill 
            д.т.н., проф. Черепанська І. Ю.\\
            \hfill (дата)\\[2cm]
            Виконавець\hfill (підпис)\hfill Погорєлов Б. Ю.\\
            \hfill (дата)\\[2cm]
        \end{flushleft}
        \centering
        2025
    \end{titlepage}
}

% \begin{document}
    % \maketitlepage{автоматизації та систем неруйнівного контролю}{ПМ-11}
    % {XXXXXXXXXXXXXXXXXXXXXXXXXXXXXXXX}{X}{XXXX}
% \end{document}

\begin{document}
    \makrosLab{7}{п}{
    Розробка алгоритмів роботи\\
    мікроконтролерів}
\section*{Тема роботи}
Розробка алгоритмів роботи мікроконтролерів

\section*{Мета роботи}
Навчитись розробляти блок-схеми алгоритмів згідно ДСТУ.

\section*{Завдання}
Варіант 9

Розробити алгоритм роботи мікроконтролера керування вологістю у
складському приміщенні



\section*{Покрокове пояснення алгоритму}
\begin{enumerate}
    \item Старт
    \item Запит в оператора конфігурацію роботи, якщо не отримано завантажити з постійно-запам'ятовуючого пристрою.
    \item Зчитування значення вологості з сенсору та зберішання в оперативно-запам'ятовуючому пристрої
    \item Якщо вологість в межах норми, відімкнути пристрої керування.
    \item Якщо вологість недостатня, ввімкнути зволожувач, інакше - осушувач.
    \item Якщо живлення присутнє, повторити вимір
    \item Кінець
\end{enumerate}

\newpage

\begin{figure}[h]
    \centering
    \includegraphics[width=0.65\textwidth]{imgs/PW7.png}
    \caption*{Рис: 7.1 Блок-схема алгоритму }
\end{figure}



\section*{Висновок}
У ході виконання практичної роботи я навчився  розробляти блок-схеми алгоритмів програми для мікроконтроллерів згідно ДСТУ.


\end{document}
\end{document}
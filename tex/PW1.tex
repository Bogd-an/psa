\documentclass[a4paper]{article}
\usepackage{listings}
% \documentclass[a4paper,14pt]{extarticle} 

% \usepackage[T2A]{fontenc}
% \usepackage[utf8]{inputenc}
% \usepackage[ukrainian]{babel}

% \usepackage{graphicx}
% \usepackage{geometry}
% \geometry{left=25mm, right=25mm, top=20mm, bottom=20mm}

% \renewcommand{\baselinestretch}{1.5}
% \setlength{\parindent}{3em}


\newcommand{\makrosTitle}[2]{
    \begin{titlepage}
        \centering
        \textbf{Міністерство освіти і науки України}\\
        \textbf{КИЇВСЬКИЙ ПОЛІТЕХНІЧНИЙ УНІВЕРССИТЕТ}\\[2cm]
        \raggedleft
        Кафедра автоматизації та систем неруйнівного контролю\\
        Група ПМ-11\\[2cm]
        \centering
        \textbf{ПРОЕКТУВАННЯ СИСТЕМ АВТОМАТИЗАЦІЇ}\\[1cm]
        \textbf{ЗВІТ З ЛАБОРАТОРНОЇ РОБОТИ №#1}\\[1cm]
        \textbf{#2}\\[3cm]
        \begin{flushleft}
            Керівник  \qquad\qquad\quad \hfill\qquad (підпис)\hfill 
            д.т.н., проф. Черепанська І. Ю.\\
            \hfill (дата)\\[2cm]
            Виконавець\hfill (підпис)\hfill Погорєлов Б. Ю.\\
            \hfill (дата)\\[2cm]
        \end{flushleft}
        \centering
        2025
    \end{titlepage}
}

% \begin{document}
    % \maketitlepage{автоматизації та систем неруйнівного контролю}{ПМ-11}
    % {XXXXXXXXXXXXXXXXXXXXXXXXXXXXXXXX}{X}{XXXX}
% \end{document}

\begin{document}
    \makrosLab{1}{п}{
Розробка системи автоматизованого\\
або автоматичного керування \\
типовим технологічним об’єктом \\
за варіантами індивідуальних завдань.
    }

\section*{Мета роботи}
Навчитися розробляти та складати структурні схеми
систем автоматичного або автоматизованого керування різними
технологічними об'єктами.

\section*{Порядок виконання роботи}
\begin{enumerate}
    \item Ознайомитись з теоретичними відомостями (пункт 1.2.)
    \item Вивчити та описати в загальному вигляді природутехнологічного об’єкту керування (ОК) і процеси, що протікають в ньому.
    \item Визначити та описати регульовані параметри, збурення і можливі дії керування.
    \item Скласти параметричну схему ОК із вказанням фізичних величин вхідних та вихідних сигналів і збурюючих впливів.
    \item Скласти структурну схему системи керування та описати її роботу в загальному вигляді.
    \item Зробити висновки по роботі та дати відповіді на контрольні питання (пункт 1.3).
    \item Оформити звіт згідно вимог (пункт 1.4).
\end{enumerate}

\newpage

\section*{Теоретичні відомості}
HACERFOne є пристроєм для розпізнавання та аналізу радіосигналів. У поєднанні з Raspberry Pi він дозволяє створити систему автоматичного керування, яка отримує сигнали, обробляє їх та приймає рішення на основі розпізнаних даних. 

Основні характеристики системи:
\begin{itemize}
    \item Прийом та аналіз радіосигналів.
    \item Фільтрація та класифікація вхідних сигналів.
    \item Використання алгоритмів машинного навчання для розпізнавання сигналів.
    \item Автоматична передача керуючих команд технологічному об'єкту.
\end{itemize}

\section*{Розрахунки}
Розрахунки проводяться відповідно до прийнятого протоколу. Основні етапи включають:
\begin{enumerate}
    \item Аналіз спектру прийнятих сигналів.
    \item Використання математичних методів для фільтрації шумів.
    \item Визначення оптимальних параметрів алгоритмів розпізнавання.
    \item Оцінка точності розпізнавання та ефективності керування.
\end{enumerate}


\section*{Висновки}
Розроблена система дозволяє автоматично розпізнавати сигнали та передавати керуючі команди. Отримані результати підтверджують ефективність використання HACERFOne та Raspberry Pi для автоматизованого керування.

\section*{Відповіді на контрольні питання}
\begin{enumerate}
    \item \textbf{Що таке система автоматичного керування?}\\
    Це система, що здійснює керування технологічним об'єктом без участі оператора.
    \item \textbf{В чому полягає процес функціонування САК?}\\
    Система аналізує вхідні сигнали, обробляє їх та формує вихідний керуючий сигнал для об'єкта.
    \item \textbf{Основні терміни:}\
    \textbf{Система} – сукупність взаємопов'язаних елементів, що виконують певну функцію.\\
    \textbf{Об’єкт} – елемент, що підлягає керуванню.\\
    \textbf{Регулятор} – пристрій, що формує керуючий вплив.\\
    \textbf{Виконавчий механізм} – пристрій, що здійснює фізичну реалізацію керуючого впливу.\\
    \textbf{Регулювальний орган} – елемент, що змінює параметри об'єкта.\\
    \textbf{ТОК} – технологічний об'єкт керування.
    \item \textbf{З яких елементів складається система автоматичного керування?}\\
    Система складається з датчиків, контролера, виконавчих механізмів та керованого об'єкта.
\end{enumerate}


\end{document}
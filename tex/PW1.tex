\documentclass[a4paper]{article}
\usepackage{listings}
\usepackage[T2A]{fontenc}
\usepackage[utf8]{inputenc}
\usepackage[ukrainian]{babel}
\usepackage{tikz}
\usepackage{lastpage} 
\usepackage[left=2.5cm, right=1.5cm, top=1.5cm, bottom=2.5cm]{geometry}
\usepackage{fancyhdr}
% \usepackage{graphicx}

\pagestyle{fancy}
\fancyhf{}
\renewcommand{\headrulewidth}{0pt}
\renewcommand{\footrulewidth}{0pt}
% \pagestyle{empty}

\newcommand{\makrosTitle}[2]{
\thispagestyle{empty}
        \centering
        \textbf{Міністерство освіти і науки України}\\
        \textbf{КИЇВСЬКИЙ ПОЛІТЕХНІЧНИЙ УНІВЕРССИТЕТ}\\[2cm]
        \raggedleft
        Кафедра автоматизації та систем неруйнівного контролю\\
        Група ПМ-11
        \vfill
        \centering
        \textbf{ПРОЕКТУВАННЯ СИСТЕМ АВТОМАТИЗАЦІЇ}\\[1cm]
        \textbf{ЗВІТ З ЛАБОРАТОРНОЇ РОБОТИ №#1}\\[1cm]
        \textbf{#2}
        \vfill
        \begin{flushleft}
            Керівник  \qquad\qquad\quad \hfill\qquad (підпис)\hfill 
            д.т.н., проф. Черепанська І. Ю.\\
            \hfill (дата)\\[2cm]
            Виконавець\hfill (підпис)\hfill Погорєлов Б. Ю.\\
            \hfill (дата)
        \end{flushleft}
        \vfill
        \centering
        2025
}

\newcommand{\makrosFrameBig}[2]{
    \thispagestyle{empty} % Вимикає номер сторінки на першій сторінці
    
    \begin{tikzpicture}[remember picture, overlay]
        \begin{scope}[shift={([xshift = 20 mm, yshift = 10 mm]current page.south west)}]
            \draw[line width=2] (0,0) rectangle (180 mm,277 mm);
        \end{scope}
    \end{tikzpicture}
    
    \begin{tikzpicture}[remember picture, overlay]
        \begin{scope}[shift={([xshift = 20 mm, yshift = 10 mm]current page.south west)}, x=1mm, y=1mm]
            \draw[line width=2] (0,0) rectangle (180,40);
            \draw[line width=2]  (7,40) -- (7, 25);
            \draw[line width=2] (17,40) -- (17, 0);
            \draw[line width=2] (40,40) -- (40, 0);
            \draw[line width=2] (55,40) -- (55, 0);
            \draw[line width=2] (65,40) -- (65, 0);
            \draw[line width=2] (135,25) -- (135,0);
            \draw[line width=2] (140,15) -- (140,20);
            \draw[line width=2] (145,15) -- (145,20);
            \draw[line width=2] (150,25) -- (150,15);
            \draw[line width=2] (165,25) -- (165,15);
        
            \draw (0,35) -- (65, 35);
            \draw[line width=2] (0,30) -- (65, 30);
            \draw[line width=2] (0,25) -- (180, 25);
            \draw (0,20) -- (65, 20);
            \draw (0,15) -- (65, 15);
            \draw (0,10) -- (65, 10);
            \draw (0,5) -- (65, 5);
        
            \draw[line width=2] (135,20) -- (180, 20);
            \draw[line width=2] (135,15) -- (180, 15);
            
            \node at (3.5, 27.5) {Зм.};
            \node at (12, 27.5) {Лист};
            \node at (28.5, 27.5) {№ докум.};
            \node at (47.5, 27.5) {Підпис};
            \node at (60, 27.5) {Дата};
            
            \node at (7, 22.5) {Розроб.};
            \node at (6.5, 17.5) {Перев.};
            \node at (8.5, 7.5) {Н. Контр.};
            \node[align=left] at (5, 2.5) {Затв.};
            
            \node at (142.5, 22.5) {Літ.};
            \node at (157.5, 22.5) {Аркуш};
            \node at (172, 22.5) {Аркушів};
        
            \node[align=left, font=\itshape, anchor=south west, scale=0.9] at (16, 20) {Погорєлов Б.Ю.};
            \node[align=left, font=\itshape, anchor=south west, scale=0.8] at (16, 15) {Черепанська І.Ю.};
            \node[align=left, font=\itshape, anchor=south west, scale=0.8] at (16, 0) {Черепанська І.Ю.};
        
            \node[anchor=center, font=\itshape, scale=1.5] at (122, 32) {#1};
            \node[align=center, font=\itshape, anchor=center] at (100, 12) {#2};
            \node[align=left, font=\itshape, anchor=south west, scale=0.9] at (135, 5) {КПІ ім. І. Сікорського, ПБФ};
            \node[anchor=center, font=\itshape] at (158, 17) {2};
            \node[anchor=center, font=\itshape] at (172, 17) {\pageref{LastPage}};    
        \end{scope} 
    \end{tikzpicture}
}

\newcommand{\makrosFrameSmall}[1]{
    % \thispagestyle{empty} % Вимикає номер сторінки на першій сторінці
    
    \begin{tikzpicture}[remember picture, overlay]
        \begin{scope}[shift={([xshift = 20 mm, yshift = 10 mm]current page.south west)}]
            \draw[line width=2] (0,0) rectangle (180 mm,277 mm);
        \end{scope}
    \end{tikzpicture}
    
    \begin{tikzpicture}[remember picture, overlay]
        \begin{scope}[shift={([xshift = 20 mm, yshift = 10 mm]current page.south west)}, x=1mm, y=1mm]
            \draw[line width=2] (0,0) rectangle (180,15);
            \draw[line width=2] (7,0) -- (7, 15);
            \draw[line width=2] (17,0) rectangle (43,15);
            \draw[line width=2] (55,0) rectangle (64,15);
            \draw[line width=2] (170,0) -- (170, 15);

            \draw[line width=2] (0,5) -- (64, 5);
            \draw               (0,10) -- (64, 10);
            \draw[line width=2] (170,8) -- (180, 8);

            \node[anchor=center, scale=0.8] at (3.5, 2.5) {Змн.};
            \node[anchor=center, scale=0.9] at (12, 2.5) {Арк.};
            \node[anchor=center] at (30, 2.5) {№~докум.};
            \node[anchor=center, scale=0.9] at (49, 2.5) {Підпис};
            \node[anchor=center, scale=0.9] at (59, 2.5) {Дата};
            \node[anchor=center, font=\itshape, scale=1.5] at (115, 7.5) 
                {#1};
            \node[anchor=center] at (175, 12) {Арк.};
            \node[anchor=center] at (175, 4) {\thepage};
            
        \end{scope}
    \end{tikzpicture}
}

% \makrosLab{1}{Шифр}{Назва}
\newcommand{\makrosLab}[3]{ 
    \fancyfoot[C]{\makrosFrameSmall{#2}}
    \makrosTitle{#1}{#3}
    \newpage
    \makrosFrameBig{#2}{#3}
    \raggedright
}


\begin{document}
    \makrosLab{1}{п}{
Розробка системи автоматизованого\\
або автоматичного керування \\
типовим технологічним об’єктом \\
за варіантами індивідуальних завдань.
    }

\section*{Мета роботи}
Навчитися розробляти та складати структурні схеми
систем автоматичного або автоматизованого керування різними
технологічними об'єктами.

\section*{Порядок виконання роботи}
\begin{enumerate}
    \item Ознайомитись з теоретичними відомостями (пункт 1.2.)
    \item Вивчити та описати в загальному вигляді природутехнологічного об’єкту керування (ОК) і процеси, що протікають в ньому.
    \item Визначити та описати регульовані параметри, збурення і можливі дії керування.
    \item Скласти параметричну схему ОК із вказанням фізичних величин вхідних та вихідних сигналів і збурюючих впливів.
    \item Скласти структурну схему системи керування та описати її роботу в загальному вигляді.
    \item Зробити висновки по роботі та дати відповіді на контрольні питання (пункт 1.3).
    \item Оформити звіт згідно вимог (пункт 1.4).
\end{enumerate}

\newpage

\section*{Теоретичні відомості}
HACERFOne є пристроєм для розпізнавання та аналізу радіосигналів. У поєднанні з Raspberry Pi він дозволяє створити систему автоматичного керування, яка отримує сигнали, обробляє їх та приймає рішення на основі розпізнаних даних. 

Основні характеристики системи:
\begin{itemize}
    \item Прийом та аналіз радіосигналів.
    \item Фільтрація та класифікація вхідних сигналів.
    \item Використання алгоритмів машинного навчання для розпізнавання сигналів.
    \item Автоматична передача керуючих команд технологічному об'єкту.
\end{itemize}

\section*{Розрахунки}
Розрахунки проводяться відповідно до прийнятого протоколу. Основні етапи включають:
\begin{enumerate}
    \item Аналіз спектру прийнятих сигналів.
    \item Використання математичних методів для фільтрації шумів.
    \item Визначення оптимальних параметрів алгоритмів розпізнавання.
    \item Оцінка точності розпізнавання та ефективності керування.
\end{enumerate}


\section*{Висновки}
Розроблена система дозволяє автоматично розпізнавати сигнали та передавати керуючі команди. Отримані результати підтверджують ефективність використання HACERFOne та Raspberry Pi для автоматизованого керування.

\section*{Відповіді на контрольні питання}
\begin{enumerate}
    \item \textbf{Що таке система автоматичного керування?}\\
    Це система, що здійснює керування технологічним об'єктом без участі оператора.
    \item \textbf{В чому полягає процес функціонування САК?}\\
    Система аналізує вхідні сигнали, обробляє їх та формує вихідний керуючий сигнал для об'єкта.
    \item \textbf{Основні терміни:}\
    \textbf{Система} – сукупність взаємопов'язаних елементів, що виконують певну функцію.\\
    \textbf{Об’єкт} – елемент, що підлягає керуванню.\\
    \textbf{Регулятор} – пристрій, що формує керуючий вплив.\\
    \textbf{Виконавчий механізм} – пристрій, що здійснює фізичну реалізацію керуючого впливу.\\
    \textbf{Регулювальний орган} – елемент, що змінює параметри об'єкта.\\
    \textbf{ТОК} – технологічний об'єкт керування.
    \item \textbf{З яких елементів складається система автоматичного керування?}\\
    Система складається з датчиків, контролера, виконавчих механізмів та керованого об'єкта.
\end{enumerate}


\end{document}
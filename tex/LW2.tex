\documentclass[a4paper]{article}
% \documentclass[a4paper,14pt]{extarticle} 

% \usepackage[T2A]{fontenc}
% \usepackage[utf8]{inputenc}
% \usepackage[ukrainian]{babel}

% \usepackage{graphicx}
% \usepackage{geometry}
% \geometry{left=25mm, right=25mm, top=20mm, bottom=20mm}

% \renewcommand{\baselinestretch}{1.5}
% \setlength{\parindent}{3em}


\newcommand{\makrosTitle}[2]{
    \begin{titlepage}
        \centering
        \textbf{Міністерство освіти і науки України}\\
        \textbf{КИЇВСЬКИЙ ПОЛІТЕХНІЧНИЙ УНІВЕРССИТЕТ}\\[2cm]
        \raggedleft
        Кафедра автоматизації та систем неруйнівного контролю\\
        Група ПМ-11\\[2cm]
        \centering
        \textbf{ПРОЕКТУВАННЯ СИСТЕМ АВТОМАТИЗАЦІЇ}\\[1cm]
        \textbf{ЗВІТ З ЛАБОРАТОРНОЇ РОБОТИ №#1}\\[1cm]
        \textbf{#2}\\[3cm]
        \begin{flushleft}
            Керівник  \qquad\qquad\quad \hfill\qquad (підпис)\hfill 
            д.т.н., проф. Черепанська І. Ю.\\
            \hfill (дата)\\[2cm]
            Виконавець\hfill (підпис)\hfill Погорєлов Б. Ю.\\
            \hfill (дата)\\[2cm]
        \end{flushleft}
        \centering
        2025
    \end{titlepage}
}

% \begin{document}
    % \maketitlepage{автоматизації та систем неруйнівного контролю}{ПМ-11}
    % {XXXXXXXXXXXXXXXXXXXXXXXXXXXXXXXX}{X}{XXXX}
% \end{document}

\begin{document}
    \makrosLab{2}{л}{
        Розробка та складання схем \\
        електричних принципових керування \\ 
        промисловими двигунами
    }

    \section*{Тема роботи}
    Розробка та складання схем електричних принципових керування
    промисловими двигунами

    \section*{Мета роботи}
    Вивчити будову та принцип дії промислових двигунів
    різних типів, як складових систем автоматичного
    керування / регулювання / контролю. Навчитися складати схеми електричні
    принципові для керування промисловими двигунами різних типів.

    \section*{Вихідні дані (Варіант 09)}
Для варіанту 9:
\begin{itemize}
    \item Номінальна потужність на валу, \( P_{\text{ном.мех}} = 125 \, \text{кВт} \)
    \item Коефіцієнт потужності, \( \cos \varphi_{\text{ном}} = 0.95 \)
    \item Номінальна швидкість обертання, \( n_{\text{ном}} = 1460 \, \text{об/хв} \)
    \item Коефіцієнт перенавантажної здатності, \( \gamma = 2.3 \)
    \item ККД, \( \eta = 91\% \)
    \item Коефіцієнт кратності пускового струму, \( \alpha = 5.1 \)
    \item Коефіцієнт кратності пускового моменту, \( \beta = 2.35 \)
\end{itemize}

    \newpage 
    
    \section*{Завдання}
Трифазний асинхронний двигун з короткозамкненим ротором має такі параметри:
\begin{enumerate}
    \item напруга живлення: $380/220$ В;
    \item номінальна потужність на валу: $P_{\text{ном.мех}}$;
    \item номінальна швидкість: $n_{\text{ном}}$;
    \item коефіцієнт корисної дії: $\eta$;
    \item коефіцієнт потужності: $\cos \varphi_{\text{ном}}$;
    \item коефіцієнт кратності пускового струму: $\alpha$;
    \item коефіцієнт кратності пускового моменту: $\beta = \frac{M_{\text{пуск}}}{M_{\text{н}}}$;
    \item коефіцієнт перенавантажної здатності: $\gamma = \frac{M_{\text{max}}}{M_{\text{н}}}$.
\end{enumerate}

Двигун увімкнено за схемою "зірка" до мережі з лінійною напругою $U_{\text{лін}} = 380$ В, частотою $f = 50$ Гц.

З врахуванням даних таблиці визначити:
\begin{enumerate}
    \item споживану потужність: активну, реактивну, повну;
    \item споживаний струм;
    \item пусковий струм;
    \item ємність конденсаторів для підвищення $\cos\varphi$ до $0,95$ при вмиканні їх за схемами "зірка" та "трикутник", побудувати векторні діаграми напруги і струмів та трикутник потужностей;
    \item обертаючі моменти двигуна: номінальний, пусковий, критичний;
    \item номінальне і критичне значення ковзання;
    \item обертаючий момент двигуна при значеннях ковзання: $S = 0$; $S_{\text{ном}}$; $0,8S_{\text{кр}}$; $S_{\text{кр}}$; $1,2S_{\text{кр}}$; $0,2$; $0,4$; $0,6$; $0,8$; $1$.
\end{enumerate}
    
    \newpage 
    \section*{Схеми}

\begin{figure}[h]
    \centering
    \includegraphics[width=0.8\textwidth]{imgs/LW2.1.png}
    \caption*{Рис. 2.1: Схема електрична принципова реверсивного керування асинхронним електродвигуном}
\end{figure} 

\begin{figure}[h]
    \centering
    \includegraphics[width=0.85\textwidth]{imgs/LW2.2.png}
    \caption*{Рис. 2.2: Схема електрична принципова нереверсивного керування асинхронним електродвигуном}
\end{figure} 
\newpage
\begin{figure}[h]
    \centering
    \includegraphics[width=0.65\textwidth]{imgs/LW2.3.png}
    \caption*{Рис. 2.3: Схема електрична принципова керування трифазним асинхронним електродвигуном з короткозамкненим ротором з обмеженням  пускового струму і моменту активними опорами}
\end{figure} 

\begin{figure}[h]
    \centering
    \includegraphics[width=0.55\textwidth]{imgs/LW2.4.png}
    \caption*{Рис. 2.4: Схема електрична принципова керування трифазним асинхронним
електродвигуном з перемиканням обмотки статора iз «зірки» на «трикутник» при пуску}
\end{figure}

\newpage
\begin{figure}[h]
    \centering
    \includegraphics[width=0.7\textwidth]{imgs/LW2.4.2.png}
    \caption*{Рис. 2.5: Схема електрична принципова}
\end{figure}



    \section*{Розрахунки}


\subsection*{1. Споживана потужність}
\textbf{Активна потужність:}
\[
P_{\text{ном.ел}} = \frac{P_{\text{ном.мех}}}{\eta} = \frac{125}{0.91} \approx 137.36 \, \text{кВт}
\]

\textbf{Повна потужність:}
\[
S_{\text{ном}} = \frac{P_{\text{ном.ел}}}{\cos \varphi_{\text{ном}}} = \frac{137.36}{0.95} \approx 144.59 \, \text{кВА}
\]

\textbf{Реактивна потужність:}
\[
Q_{\text{ном}} = \sqrt{S_{\text{ном}}^2 - P_{\text{ном.ел}}^2} = \sqrt{144.59^2 - 137.36^2} \approx 45.16 \, \text{кВАр}
\]

\subsection*{2. Споживаний струм}
\[
I_{л} = \frac{S_{\text{ном}}}{\sqrt{3} \cdot U_{\text{лін}}} = \frac{144.59 \times 10^3}{\sqrt{3} \cdot 380} \approx 219.6 \, \text{А}
\]

\subsection*{3. Пусковий струм}
\[
I_{\text{пуск}} = \alpha \cdot I_{\text{ном}} = 5.1 \times 219.6 \approx 1120 \, \text{А}
\]

\subsection*{4. Ємність конденсаторів для підвищення \( \cos \varphi \) до 0.95}
\textbf{Для схеми "зірка":}
\[
C_Y = \frac{Q_{\text{конд}}}{2 \pi f \cdot 3 U_{ф}^2} = \frac{45.16 \times 10^3}{2 \pi \cdot 50 \cdot 3 \cdot (220)^2} \approx 49.5 \, \mu\text{F}
\]

\textbf{Для схеми "трикутник":}
\[
C_{\Delta} = \frac{Q_{\text{конд}}}{2 \pi f \cdot 3 U_{\text{лін}}^2} = \frac{45.16 \times 10^3}{2 \pi \cdot 50 \cdot 3 \cdot (380)^2} \approx 16.5 \, \mu\text{F}
\]

\subsection*{5. Обертові моменти}
\textbf{Номінальний момент:}
\[
M_{\text{ном}} = \frac{P_{\text{ном.мех}}}{\Omega} = \frac{125 \times 10^3}{2 \pi \cdot \frac{1460}{60}} \approx 818.5 \, \text{Нм}
\]

\textbf{Пусковий момент:}
\[
M_{\text{пуск}} = \beta \cdot M_{\text{ном}} = 2.35 \times 818.5 \approx 1923.5 \, \text{Нм}
\]

\textbf{Критичний момент:}
\[
M_{\text{кр}} = \gamma \cdot M_{\text{ном}} = 2.3 \times 818.5 \approx 1882.6 \, \text{Нм}
\]

\subsection*{6. Ковзання}
\textbf{Номінальне ковзання:}
\[
s_{\text{ном}} = \frac{n_1 - n_{\text{ном}}}{n_1} = \frac{1500 - 1460}{1500} \approx 0.0267
\]

\textbf{Критичне ковзання:}
\[
s_{\text{кр}} = s_{\text{ном}} \cdot \left( \gamma + \sqrt{\gamma^2 - 1} \right) = 0.0267 \cdot \left( 2.3 + \sqrt{2.3^2 - 1} \right) \approx 0.12
\]

\subsection*{7. Потужність втрат}
\textbf{Втрати в обмотках:}
\[
P_{\text{втр}} = P_{\text{ном.ел}} - P_{\text{ном.мех}} = 137.36 - 125 = 12.36 \, \text{кВт}
\]



\section*{Обчислення обертаючого моменту}

Обертаючий момент асинхронного двигуна визначається за формулою:

\begin{equation}
    M = \frac{M_{\text{кр}}}{\frac{S_{\text{кр}}}{S} + \frac{S}{S_{\text{кр}}}}
\end{equation}

де:
\begin{itemize}
    \item $M$ — обертаючий момент двигуна при ковзанні $S$, Н·м;
    \item $M_{\text{кр}}$ — критичний момент (максимальний обертаючий момент), Н·м;
    \item $S$ — значення ковзання;
    \item $S_{\text{кр}}$ — критичне ковзання (значення ковзання, при якому досягається максимальний момент).
\end{itemize}

Ця формула дозволяє обчислити момент для різних значень ковзання:
\begin{itemize}
    \item при $S = 0$ момент теоретично дорівнює нулю;
    \item при $S = S_{\text{кр}}$ двигун розвиває максимальний момент $M_{\text{кр}}$;
    \item при великих значеннях $S$ момент зменшується.
\end{itemize}

\begin{figure}[h]
    \centering
    \includegraphics[width=1\textwidth]{imgs/LW2.5.png}
    \caption*{Рис. 2.6: Розрахунок крутного моменту}
\end{figure} 


\section*{Висновки}

Отримані результати дозволяють оцінити параметри роботи трифазного асинхронного двигуна, його енергетичні характеристики та вибір необхідних ємностей для підвищення коефіцієнта потужності.

\section*{Контрольні питання}
\begin{enumerate}
    \item Чому асинхронний двигун так називається? \\
    Асинхронний двигун називається так тому, що частота обертання його ротора не співпадає з частотою обертання магнітного поля статора (яка визначається частотою змінного струму). Різниця між цими частотами називається ковзанням.
    
    \item Чому є небажаною велика сила пускового струму? \\
    Велика сила пускового струму небажана, оскільки вона може призвести до значних механічних та електричних навантажень на двигун і мережу, викликати пошкодження ізоляції проводів, зменшити термін служби обладнання, а також викликати перевантаження трансформаторів і підстанцій.
    
    \item Що використовують для зниження сили пускового струму? \\
    Для зниження сили пускового струму використовують спеціальні пристрої, такі як стартери з обмеженням струму, трансформатори з регульованим напругою або пристрої плавного пуску, що забезпечують поступове збільшення напруги на двигуні.
\end{enumerate}


\end{document}
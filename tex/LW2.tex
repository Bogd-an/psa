\documentclass[a4paper]{article}
% \documentclass[a4paper,14pt]{extarticle} 

% \usepackage[T2A]{fontenc}
% \usepackage[utf8]{inputenc}
% \usepackage[ukrainian]{babel}

% \usepackage{graphicx}
% \usepackage{geometry}
% \geometry{left=25mm, right=25mm, top=20mm, bottom=20mm}

% \renewcommand{\baselinestretch}{1.5}
% \setlength{\parindent}{3em}


\newcommand{\makrosTitle}[2]{
    \begin{titlepage}
        \centering
        \textbf{Міністерство освіти і науки України}\\
        \textbf{КИЇВСЬКИЙ ПОЛІТЕХНІЧНИЙ УНІВЕРССИТЕТ}\\[2cm]
        \raggedleft
        Кафедра автоматизації та систем неруйнівного контролю\\
        Група ПМ-11\\[2cm]
        \centering
        \textbf{ПРОЕКТУВАННЯ СИСТЕМ АВТОМАТИЗАЦІЇ}\\[1cm]
        \textbf{ЗВІТ З ЛАБОРАТОРНОЇ РОБОТИ №#1}\\[1cm]
        \textbf{#2}\\[3cm]
        \begin{flushleft}
            Керівник  \qquad\qquad\quad \hfill\qquad (підпис)\hfill 
            д.т.н., проф. Черепанська І. Ю.\\
            \hfill (дата)\\[2cm]
            Виконавець\hfill (підпис)\hfill Погорєлов Б. Ю.\\
            \hfill (дата)\\[2cm]
        \end{flushleft}
        \centering
        2025
    \end{titlepage}
}

% \begin{document}
    % \maketitlepage{автоматизації та систем неруйнівного контролю}{ПМ-11}
    % {XXXXXXXXXXXXXXXXXXXXXXXXXXXXXXXX}{X}{XXXX}
% \end{document}

\begin{document}
    \makrosLab{2}{л}{
        Розробка та складання схем \\
        електричних принципових керування \\ 
        промисловими двигунами
    }

    \section*{Тема роботи}
    Розробка та складання схем електричних принципових керування
    промисловими двигунами

    \section*{Мета роботи}
    Вивчити будову та принцип дії промислових двигунів
    різних типів, як складових систем автоматичного
    керування / регулювання / контролю. Навчитися складати схеми електричні
    принципові для керування промисловими двигунами різних типів.

    \section*{Вихідні дані (Варіант 09)}
    \begin{table}[h!]
        \centering
        \begin{tabular}{|l|c|}
            \hline
            \textbf{Параметр} & \textbf{Значення} \\
            \hline
            Потужність, кВт & 1,0 \\
            \hline
            cos$\varphi$ & 0,86 \\
            \hline
            Швидкість обертання n ном, об/хв & 2850 \\
            \hline
            $\gamma$ (перенавантажувальна здатність) & 2,2 \\
            \hline
            ККД, \% & 91 \\
            \hline
            $\alpha$ (кратність пускового струму) & 5,1 \\
            \hline
            $\beta$ (кратність пускового моменту) & 2,35 \\
            \hline
        \end{tabular}
        \caption{Вихідні дані для розрахунків}
    \end{table}

    \newpage 
    
    \section*{Завдання}
Трифазний асинхронний двигун з короткозамкненим ротором має такі параметри:
\begin{enumerate}
    \item напруга живлення: $380/220$ В;
    \item номінальна потужність на валу: $P_{\text{ном.мех}}$;
    \item номінальна швидкість: $n_{\text{ном}}$;
    \item коефіцієнт корисної дії: $\eta$;
    \item коефіцієнт потужності: $\cos \varphi_{\text{ном}}$;
    \item коефіцієнт кратності пускового струму: $\alpha$;
    \item коефіцієнт кратності пускового моменту: $\beta = \frac{M_{\text{пуск}}}{M_{\text{н}}}$;
    \item коефіцієнт перенавантажної здатності: $\gamma = \frac{M_{\text{max}}}{M_{\text{н}}}$.
\end{enumerate}

Двигун увімкнено за схемою "зірка" до мережі з лінійною напругою $U_{\text{лін}} = 380$ В, частотою $f = 50$ Гц.

З врахуванням даних таблиці визначити:
\begin{enumerate}
    \item споживану потужність: активну, реактивну, повну;
    \item споживаний струм;
    \item пусковий струм;
    \item ємність конденсаторів для підвищення $\cos\varphi$ до $0,95$ при вмиканні їх за схемами "зірка" та "трикутник", побудувати векторні діаграми напруги і струмів та трикутник потужностей;
    \item обертаючі моменти двигуна: номінальний, пусковий, критичний;
    \item номінальне і критичне значення ковзання;
    \item обертаючий момент двигуна при значеннях ковзання: $S = 0$; $S_{\text{ном}}$; $0,8S_{\text{кр}}$; $S_{\text{кр}}$; $1,2S_{\text{кр}}$; $0,2$; $0,4$; $0,6$; $0,8$; $1$.
\end{enumerate}
    
    \newpage 
    \section*{Схеми}

\begin{figure}[h]
    \centering
    \includegraphics[width=0.45\textwidth]{imgs/LW2.1.png}
    \caption*{Рис. 2.1: Схема електрична принципова реверсивного керування асинхронним електродвигуном}
\end{figure} 

\begin{figure}[h]
    \centering
    \includegraphics[width=0.5\textwidth]{imgs/LW2.2.png}
    \caption*{Рис. 2.2:Схема електрична принципова нереверсивного керування асинхронним електродвигуном}
\end{figure} 

\begin{figure}[h]
    \centering
    \includegraphics[width=0.5\textwidth]{imgs/LW2.3.png}
    \caption*{Рис. 2.3: Схема електрична принципова керування трифазним асинхронним електродвигуном з короткозамкненим ротором з обмеженням  пускового струму і моменту активними опорами}
\end{figure} 

    \section*{Розрахунки}

\subsection*{Споживана потужність}
\begin{align*}
    P_{\text{спож}} &= \frac{P_{\text{ном.мех}}}{\eta} = \frac{1,0 \text{ кВт}}{0,91} = 1,098 \text{ кВт}.
\end{align*}

\begin{align*}
    S &= \frac{P_{\text{спож}}}{\cos\varphi} = \frac{1,098}{0,86} = 1,277 \text{ кВА}.
\end{align*}

\begin{align*}
    Q &= \sqrt{S^2 - P_{\text{спож}}^2} = \sqrt{1,277^2 - 1,098^2} = 0,635 \text{ кВАр}.
\end{align*}

\subsection*{Споживаний струм}

\begin{align*}
    I &= \frac{S}{\sqrt{3} U_{\text{лін}}} = \frac{1,277 \times 10^3}{\sqrt{3} \times 380} = 1,942 \text{ А}.
\end{align*}

\subsection*{Пусковий струм}

\begin{align*}
    I_{\text{пуск}} &= \alpha I = 5,1 \times 1,942 = 9,9 \text{ А}.
\end{align*}

\subsection*{Обертаючі моменти}

\begin{align*}
    M_{\text{н}} &= \frac{P_{\text{ном.мех}} \times 9550}{n_{\text{ном}}} = \frac{1,0 \times 9550}{2850} = 3,35 \text{ Нм}.
\end{align*}

\begin{align*}
    M_{\text{пуск}} &= \beta M_{\text{н}} = 2,35 \times 3,35 = 7,87 \text{ Нм}.
\end{align*}

\begin{align*}
    M_{\text{кр}} &= \gamma M_{\text{н}} = 2,2 \times 3,35 = 7,37 \text{ Нм}.
\end{align*}

\subsection*{Ковзання}

\begin{align*}
    S_{\text{ном}} &= \frac{n_{\text{с}} - n_{\text{ном}}}{n_{\text{с}}} \approx \frac{3000 - 2850}{3000} = 0,05.
\end{align*}

\begin{align*}
    S_{\text{кр}} &= \frac{M_{\text{н}}}{M_{\text{кр}}} = \frac{3,35}{7,37} = 0,455.
\end{align*}

\subsection*{Ємність конденсаторів}

\begin{align*}
    Q_{\text{кор}} &= P_{\text{спож}} (\tan \varphi_{\text{поч}} - \tan \varphi_{\text{кінц}}) = 1,098 (\tan 30^\circ - \tan 18^\circ) \approx 0,256 \text{ кВАр}.
\end{align*}

\begin{align*}
    C &= \frac{Q_{\text{кор}}}{2 \pi f U^2} = \frac{0,256 \times 10^3}{2 \pi \times 50 \times (380)^2} = 5,7 \text{ мкФ}.
\end{align*}

\section*{Висновки}

Отримані результати дозволяють оцінити параметри роботи трифазного асинхронного двигуна, його енергетичні характеристики та вибір необхідних ємностей для підвищення коефіцієнта потужності.

\section*{Контрольні питання}
\begin{enumerate}
    \item Чому асинхронний двигун так називається? \\
    Асинхронний двигун називається так тому, що частота обертання його ротора не співпадає з частотою обертання магнітного поля статора (яка визначається частотою змінного струму). Різниця між цими частотами називається ковзанням.
    
    \item Чому є небажаною велика сила пускового струму? \\
    Велика сила пускового струму небажана, оскільки вона може призвести до значних механічних та електричних навантажень на двигун і мережу, викликати пошкодження ізоляції проводів, зменшити термін служби обладнання, а також викликати перевантаження трансформаторів і підстанцій.
    
    \item Що використовують для зниження сили пускового струму? \\
    Для зниження сили пускового струму використовують спеціальні пристрої, такі як стартери з обмеженням струму, трансформатори з регульованим напругою або пристрої плавного пуску, що забезпечують поступове збільшення напруги на двигуні.
\end{enumerate}


\end{document}
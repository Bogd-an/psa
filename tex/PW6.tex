\documentclass[a4paper]{article}
\usepackage{listings}
% \documentclass[a4paper,14pt]{extarticle} 

% \usepackage[T2A]{fontenc}
% \usepackage[utf8]{inputenc}
% \usepackage[ukrainian]{babel}

% \usepackage{graphicx}
% \usepackage{geometry}
% \geometry{left=25mm, right=25mm, top=20mm, bottom=20mm}

% \renewcommand{\baselinestretch}{1.5}
% \setlength{\parindent}{3em}


\newcommand{\makrosTitle}[2]{
    \begin{titlepage}
        \centering
        \textbf{Міністерство освіти і науки України}\\
        \textbf{КИЇВСЬКИЙ ПОЛІТЕХНІЧНИЙ УНІВЕРССИТЕТ}\\[2cm]
        \raggedleft
        Кафедра автоматизації та систем неруйнівного контролю\\
        Група ПМ-11\\[2cm]
        \centering
        \textbf{ПРОЕКТУВАННЯ СИСТЕМ АВТОМАТИЗАЦІЇ}\\[1cm]
        \textbf{ЗВІТ З ЛАБОРАТОРНОЇ РОБОТИ №#1}\\[1cm]
        \textbf{#2}\\[3cm]
        \begin{flushleft}
            Керівник  \qquad\qquad\quad \hfill\qquad (підпис)\hfill 
            д.т.н., проф. Черепанська І. Ю.\\
            \hfill (дата)\\[2cm]
            Виконавець\hfill (підпис)\hfill Погорєлов Б. Ю.\\
            \hfill (дата)\\[2cm]
        \end{flushleft}
        \centering
        2025
    \end{titlepage}
}

% \begin{document}
    % \maketitlepage{автоматизації та систем неруйнівного контролю}{ПМ-11}
    % {XXXXXXXXXXXXXXXXXXXXXXXXXXXXXXXX}{X}{XXXX}
% \end{document}

\begin{document}
    \makrosLab{6}{п}{Побудова блок-схем алгоритмів}
\section*{Тема роботи}
Побудова блок-схем алгоритмів

\section*{Мета роботи}
Навчитись розробляти блок-схеми алгоритмів згідно ДСТУ.

\section*{Завдання}
Варіант 9

Скласти блок-схему алгоритму обчислення значень змінної згідно варіанту



\section*{Система рівнянь}
Функція \( Y \) визначається як:
\begin{equation*}
    Y = \begin{cases} 
        0.5\cos x + 4x, & x \leq 1; \\ 
        0.25x^4 + 2x^2, & x < 0; \\
        0.9\sqrt{x} - 0.8x, & x > 1.
    \end{cases}
\end{equation*}
\begin{equation*}
    x = 1.7 - e^{0.35}
\end{equation*}

% \newpage

\section*{Покрокове пояснення алгоритму}
\begin{enumerate}
    \item Введення вхідних даних:
    \begin{itemize}
        \item Отримуємо значення змінної \( x \).
        \item Якщо \( x \) не задано, обчислюємо його за формулою: \( x = 1.7 - e^{0.35} \).
    \end{itemize}
    \item Вибір відповідної гілки обчислення:
    \begin{itemize}
        \item Якщо \( x < 0 \), використовуємо формулу: \( Y = 0.25x^4 + 2x^2 \).
        \item Якщо \( x > 1 \), використовуємо формулу: \( Y = 0.9\sqrt{x} - 0.8x \).
        \item Інакше, використовуємо формулу: \( Y = 0.5 \cos x + 4x \).
    \end{itemize}
    \newpage
    \item Обчислення значення функції \( Y \):
    \begin{itemize}
        \item Виконуємо підстановку значення \( x \) у відповідну формулу.
        \item Обчислюємо результат.
    \end{itemize}
    \item Виведення результату \( Y \).
\end{enumerate}

% ```mermaid
% graph TD;
%     A(Початок) --> B{Чи задано x?};
%     B -- Так --> C[Отримати x];
%     B -- Ні --> D[Обчислити x = 1.7 - e^0.35];
%     C --> E{Яке значення x?};
%     D --> E;
%     E -- x ≤ 1 --> F[Y = 0.5 cos(x) + 4x];
%     E -- x < 0 --> G[Y = 0.25x^4 + 2x^2];
%     E -- x > 1 --> H[Y = 0.9√x - 0.8x];
%     F --> I(Вивести Y);
%     G --> I;
%     H --> I;
%     I --> J(Кінець);
% ```

\begin{figure}[h]
    \centering
    \includegraphics[width=0.8\textwidth]{imgs/PW6.png}
    \caption*{Рис: 6.1 Блок-схема алгоритму рівняння}
\end{figure}

\section*{Висновок}
У ході виконання практичної роботи я навчився  розробляти блок-схеми алгоритмів згідно ДСТУ.


\end{document}
\documentclass[a4paper]{article}
\usepackage{listings}
% \documentclass[a4paper,14pt]{extarticle} 

% \usepackage[T2A]{fontenc}
% \usepackage[utf8]{inputenc}
% \usepackage[ukrainian]{babel}

% \usepackage{graphicx}
% \usepackage{geometry}
% \geometry{left=25mm, right=25mm, top=20mm, bottom=20mm}

% \renewcommand{\baselinestretch}{1.5}
% \setlength{\parindent}{3em}


\newcommand{\makrosTitle}[2]{
    \begin{titlepage}
        \centering
        \textbf{Міністерство освіти і науки України}\\
        \textbf{КИЇВСЬКИЙ ПОЛІТЕХНІЧНИЙ УНІВЕРССИТЕТ}\\[2cm]
        \raggedleft
        Кафедра автоматизації та систем неруйнівного контролю\\
        Група ПМ-11\\[2cm]
        \centering
        \textbf{ПРОЕКТУВАННЯ СИСТЕМ АВТОМАТИЗАЦІЇ}\\[1cm]
        \textbf{ЗВІТ З ЛАБОРАТОРНОЇ РОБОТИ №#1}\\[1cm]
        \textbf{#2}\\[3cm]
        \begin{flushleft}
            Керівник  \qquad\qquad\quad \hfill\qquad (підпис)\hfill 
            д.т.н., проф. Черепанська І. Ю.\\
            \hfill (дата)\\[2cm]
            Виконавець\hfill (підпис)\hfill Погорєлов Б. Ю.\\
            \hfill (дата)\\[2cm]
        \end{flushleft}
        \centering
        2025
    \end{titlepage}
}

% \begin{document}
    % \maketitlepage{автоматизації та систем неруйнівного контролю}{ПМ-11}
    % {XXXXXXXXXXXXXXXXXXXXXXXXXXXXXXXX}{X}{XXXX}
% \end{document}
\usepackage[absolute,overlay]{textpos}

\begin{document}
    \makrosLab{3}{п}{
        Розробка функціональних схем, \\
        специфікацій, фунуціональних схем\\
         систем автоматизації. 
    }
\section*{Тема роботи}
Розробка функціональних схем систем автоматизації. Складання
специфікацій устаткування, виробів і матеріалів функціональних схем
автоматизації.

\section*{Мета роботи}
Навчитися розробляти функціональні схеми автоматизації
систем автоматичного керування, контролю або регулювання різними
технологічними об'єктами

% \section*{Порядок виконання роботи}

% \begin{enumerate}
%     \item Ознайомитись з теоретичними відомостями (пункт 3.2.)
%     \item Згідно порядку побудови та оформлення ФСА (пункт 3.3) скласти схему автоматизації за варіантом індивідуального завдання.
%     \item Зробити висновки по роботі та дати відповіді на контрольні питання (пункт 3.4).
%     \item Оформити звіт згідно вимог.
% \end{enumerate}

\section*{Висновок}
У ході виконання роботи було розроблено функціональну схему автоматизації, складено специфікацію устаткування, виробів і матеріалів. Отримано практичні навички побудови та оформлення функціональних схем автоматизації, а також оформлення звіту відповідно до вимог.

\section*{Контрольні питання}
1. Для чого призначені ФСА? — Для відображення функціональних зв'язків між елементами систем автоматизації.  

2. Яким чином на ФСА зображують технологічне устаткування? — У вигляді умовних графічних позначень.  

3. Які вимоги висуваються до графічних умовних позначень ТЗА? — Відповідність стандартам і чіткість зображення.  

4. Які вимоги висуваються до умовних позначень трубопровідних комунікацій? — Відповідність стандартам і чіткість ліній.  

5. Які вимоги до товщини ліній зв’язку на ФСА? — Відповідність стандартам і однакова товщина.  

6. Які вимоги до товщини ліній технологічне устаткування на ФСА? — Відповідність стандартам і чіткість зображення.  

7. Які вимоги до позиційних позначень ТЗА на ФСА? — Унікальність і відповідність стандартам.  

\newpage
\fancyfoot[C]{}

\begin{textblock*}{\paperwidth}(-5mm, 0mm)
    \rotatebox{90}{\includegraphics[width=\paperheight,height=\paperwidth]{imgs/PW3.1.pdf}}
\end{textblock*}

\end{document}
\documentclass[a4paper,12pt]{article}
% \documentclass[a4paper,14pt]{extarticle} 

% \usepackage[T2A]{fontenc}
% \usepackage[utf8]{inputenc}
% \usepackage[ukrainian]{babel}

% \usepackage{graphicx}
% \usepackage{geometry}
% \geometry{left=25mm, right=25mm, top=20mm, bottom=20mm}

% \renewcommand{\baselinestretch}{1.5}
% \setlength{\parindent}{3em}


\newcommand{\makrosTitle}[2]{
    \begin{titlepage}
        \centering
        \textbf{Міністерство освіти і науки України}\\
        \textbf{КИЇВСЬКИЙ ПОЛІТЕХНІЧНИЙ УНІВЕРССИТЕТ}\\[2cm]
        \raggedleft
        Кафедра автоматизації та систем неруйнівного контролю\\
        Група ПМ-11\\[2cm]
        \centering
        \textbf{ПРОЕКТУВАННЯ СИСТЕМ АВТОМАТИЗАЦІЇ}\\[1cm]
        \textbf{ЗВІТ З ЛАБОРАТОРНОЇ РОБОТИ №#1}\\[1cm]
        \textbf{#2}\\[3cm]
        \begin{flushleft}
            Керівник  \qquad\qquad\quad \hfill\qquad (підпис)\hfill 
            д.т.н., проф. Черепанська І. Ю.\\
            \hfill (дата)\\[2cm]
            Виконавець\hfill (підпис)\hfill Погорєлов Б. Ю.\\
            \hfill (дата)\\[2cm]
        \end{flushleft}
        \centering
        2025
    \end{titlepage}
}

% \begin{document}
    % \maketitlepage{автоматизації та систем неруйнівного контролю}{ПМ-11}
    % {XXXXXXXXXXXXXXXXXXXXXXXXXXXXXXXX}{X}{XXXX}
% \end{document}
\usepackage[utf8]{inputenc}
\usepackage{graphicx}
\usepackage{amsmath, amssymb}
\usepackage{geometry}
\usepackage{tikz}
\usetikzlibrary{circuits.ee.IEC, positioning}

\geometry{left=2cm,right=2cm,top=2cm,bottom=2cm}

\title{Розробка функціональних схем систем автоматизації}

\begin{document}

\maketitle

\section{Вступ}
Функціональна схема автоматизації (ФСА) є основним технічним документом, що визначає структуру вузлів системи автоматичного контролю та регулювання. Вона включає в себе технологічне устаткування, датчики, виконавчі механізми та засоби автоматизації.

У цій роботі розглядається розробка функціональної схеми для пристрою аналізу сигналів на основі HackRF One і Raspberry Pi. Цей пристрій дозволяє здійснювати аналіз та обробку радіочастотних сигналів у реальному часі, що є важливим для досліджень у сфері бездротового зв'язку, спектрального моніторингу та радіотехнічної розвідки.

\section{Методологія розробки ФСА}
Побудова функціональної схеми включає наступні етапи:
\begin{enumerate}
    \item Визначення технологічного об'єкта керування.
    \item Вибір технічних засобів автоматизації (ТЗА).
    \item Графічне зображення технологічного процесу.
    \item Розміщення умовних позначень ТЗА.
    \item Оформлення схеми згідно ДСТУ.
\end{enumerate}

\section{Функціональна схема автоматизації пристрою аналізу сигналів}

\begin{figure}[h]
    \centering
    \begin{tikzpicture}
        % HackRF One
        \node (hackrf) [draw, rectangle, minimum width=2cm, minimum height=1cm] {HackRF One};
        
        % Raspberry Pi
        \node (rpi) [draw, rectangle, minimum width=2cm, minimum height=1cm, right=of hackrf] {Raspberry Pi};
        
        % Виведення даних
        \node (display) [draw, rectangle, minimum width=2cm, minimum height=1cm, right=of rpi] {Дисплей};
        
        % Зв'язки
        \draw[->] (hackrf.east) -- (rpi.west) node[midway, above] {Сигнал};
        \draw[->] (rpi.east) -- (display.west) node[midway, above] {Обробка даних};
    \end{tikzpicture}
    \caption{Функціональна схема пристрою аналізу сигналів}
    \label{fig:signal_analysis}
\end{figure}

\section{Висновки}
Функціональна схема автоматизації дозволяє наочно представити систему контролю та керування технологічним процесом, визначити необхідні технічні засоби та їхні взаємозв'язки. 

Розглянутий пристрій на основі HackRF One і Raspberry Pi забезпечує можливість аналізу широкого спектра радіочастотних сигналів, що відкриває перспективи для його використання у сфері бездротових технологій, безпеки та моніторингу ефіру.

\end{document}
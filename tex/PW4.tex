\documentclass[a4paper]{article}
\usepackage{listings}
% \documentclass[a4paper,14pt]{extarticle} 

% \usepackage[T2A]{fontenc}
% \usepackage[utf8]{inputenc}
% \usepackage[ukrainian]{babel}

% \usepackage{graphicx}
% \usepackage{geometry}
% \geometry{left=25mm, right=25mm, top=20mm, bottom=20mm}

% \renewcommand{\baselinestretch}{1.5}
% \setlength{\parindent}{3em}


\newcommand{\makrosTitle}[2]{
    \begin{titlepage}
        \centering
        \textbf{Міністерство освіти і науки України}\\
        \textbf{КИЇВСЬКИЙ ПОЛІТЕХНІЧНИЙ УНІВЕРССИТЕТ}\\[2cm]
        \raggedleft
        Кафедра автоматизації та систем неруйнівного контролю\\
        Група ПМ-11\\[2cm]
        \centering
        \textbf{ПРОЕКТУВАННЯ СИСТЕМ АВТОМАТИЗАЦІЇ}\\[1cm]
        \textbf{ЗВІТ З ЛАБОРАТОРНОЇ РОБОТИ №#1}\\[1cm]
        \textbf{#2}\\[3cm]
        \begin{flushleft}
            Керівник  \qquad\qquad\quad \hfill\qquad (підпис)\hfill 
            д.т.н., проф. Черепанська І. Ю.\\
            \hfill (дата)\\[2cm]
            Виконавець\hfill (підпис)\hfill Погорєлов Б. Ю.\\
            \hfill (дата)\\[2cm]
        \end{flushleft}
        \centering
        2025
    \end{titlepage}
}

% \begin{document}
    % \maketitlepage{автоматизації та систем неруйнівного контролю}{ПМ-11}
    % {XXXXXXXXXXXXXXXXXXXXXXXXXXXXXXXX}{X}{XXXX}
% \end{document}
\usepackage[absolute,overlay]{textpos}

\begin{document}
    \makrosLab{4}{п}{
Вивчення правил та стандартів \\
побудови  електричних принципових\\
схем систем керування. Розробка\\
переліку елементів схеми \\
електричної принципової
    }
\section*{Тема роботи}
Розробка
переліку елементів електричної принципової схеми .

\section*{Мета роботи}
Розробка електричної принципової схеми для заданого варіанту, визначення номіналів компонентів та підготовка відповідного переліку елементів.

\section*{Завдання}

\begin{enumerate}
    \item Дослідити теоретичний лекційний матеріал.
    \item Розробити перелік елементів згідно варіанту індивідуального завдання.
    \item Заповнити основний напис
    \item За варіантом індивідуального завдання накреслити схему електичну принципову та оформити перелік елементів дотримуючись вимог ЄСКД
    \item Оформити звіт
\end{enumerate}

\section*{Електрична схема}
На рисунку наведена схема, розроблена для варіанту 9.

\begin{figure}[h]
    \centering
    \includegraphics[width=1\textwidth]{imgs/PW4.1.png}
    \caption*{Рис: 4.1}
\end{figure}

\newpage

\section*{Список використаних компонентів}
\subsection*{Резистори АП-ШК.434.110.011ТУ}
\begin{enumerate}
    \item $112$ C1-4-0,125H-2,7кОм ±10\%
    \item $118$ C1-4-0,125H-6,2кОм ±10\%
    \item $120$ C1-4-0,125H-10кОм ±10\%
    \item $122$ C1-4-0,125H-56кОм ±10\%
    \item $125$ C1-4-0,125H-820кОм ±10\%
\end{enumerate} 

\subsection*{Діоди}
\begin{enumerate}
    \item $229$ Стабілітрон КС156А (СМ3.362.812 ТУ)
\end{enumerate}

\subsection*{Транзистори}
\begin{enumerate}
    \item $330$ КТ315Б NPN (ЖКЗ.365.200ТУ)
    \item $331$ КП301Б NPN (ЖКЗ.365.220ТУ)
\end{enumerate}

\section*{Електрична принципова схема}
\begin{figure}[h]
    \centering
    \includegraphics[width=0.8\textwidth]{imgs/PW4.2.png}
    \caption*{Рис: 4.2 Електричні компоненти}
\end{figure}


\section*{Висновки}
У ході роботи було підібрано необхідні компоненти та складено їх перелік відповідно до варіанту 9 принципової схеми електричного кола.

\newpage
\fancyfoot[C]{}

\begin{textblock*}{\paperwidth}(-5mm, 0mm)
    \rotatebox{90}{\includegraphics[width=\paperheight,height=\paperwidth]{imgs/PW4.drawio.png}}
\end{textblock*}

\end{document}